\documentclass[12pt,a4paper]{article} % Añadí a4paper para estándar
\usepackage[utf8]{inputenc}
\usepackage[spanish, es-tabla]{babel} % es-tabla evita que diga "Cuadro" en vez de "Tabla"
\usepackage[margin=2.5cm]{geometry}
\usepackage{graphicx}
\usepackage{amsmath} % Para matemáticas avanzadas
\usepackage{amsfonts} % Para fuentes matemáticas

\title{Informe: PC como Workstation Científica}
\author{Simón Tovar V-31.678.578\\ Gabriel Becerra V-31.654.243}
\date{Enero 2026}

\begin{document}

\maketitle

\tableofcontents
\newpage

\section{Objetivo}
Se diseño una PC para su uso como Workstation Cientifica/STEM (Science (Ciencia), Technology (Tecnología), Engineering (Ingeniería) e Mathematics (Matemáticas)),
por lo que se busco la eficiencia para álgebra lineal avanzada, simulaciones físicas o datos en MATLAB/Python.

\section{Componentes}
\paragraph{Presupuesto} 3000\$

\begin{table}[h]
    \centering
    \makebox[\textwidth][c]{
    \begin{tabular}{|l|c|r|}
    \hline 
     & Componente & Precio \\ \hline
    Motherboard & ASRock X80 Taichi Creator ATX AM5 & 319,99\$ \\ \hline
    CPU & AMD Ryzen 9 9950X 4.3 GHz 16-Core & 549,00\$ \\ \hline
    CPU Cooler & Noctua NH-D15 82.52 CFM CPU Cooler & 139,94\$ \\ \hline
    RAM & $2 \times \text{Kingston Server Premier 32GB DDR5-5600 CL46}$ & 762,48\$ \\ \hline
    Almacenamiento & Kingston KC3000 2.048 TB M.2-2280 PCIe 4.0 X4 NVME SSD & 229,99\$ \\ \hline
    GPU & Gigabyte WINDFORCE OC SFF GeForce RTX 5070 Ti 16 GB & 799,99\$ \\ \hline
    Alimentación & Corsair RM850e (2025) 850 W Fully Modular ATX & 109,99\$ \\ \hline
    Case & Cooler Master Elite 502 ATX Mid Tower Case & 88,99\$ \\ \hline
    \end{tabular}
    }
\end{table}

\paragraph{Total:} 3000,37\$

\subsection{Captura de pantalla del precio de los componentes.}

\begin{figure}[h] % [h] intenta poner la imagen "Aquí" (Here)
    \centering    % Centra la imagen horizontalmente
    \includegraphics[width=1\textwidth]{PreciosComponentes.jpeg} % Ajusta el tamaño al 70% del ancho del texto
    \caption{Precios de los componentes 07/01/2026} % El título de la imagen
    \label{fig:precios} % Una etiqueta interna para referenciarla después
\end{figure}

\section{Justificación}

\subsection{CPU:}
AMD Ryzen 9 9950X 16 núcleos y 32 hilos, se eligió por el líder actual en 
rendimiento multihilo para plataformas de escritorio. Su arquitectura Zen 5 
soporta instrucciones AVX-512 completas, esto significa que puede realizar
hasta 16 operaciones matemáticas en un solo ciclo, esto hace que en 
simulaciones de fluidos, álgebra lineal o criptografia, el rendimiento no 
solo mejore un poco, sino que en algunos casos puede hasta duplicarse.

Ademas se eligió por sobre las opciones actuales de intel debido a que estas
ultimas utilizan una arquitectura híbrida de núcleos de ``eficiencia'' y
núcleos de ``Alto rendimiento'', por lo que no son compatibles con instrucciones AVX-512.

\subsection{Memoria RAM:}
64 GB Kigston Server Premier (ECC UDIMM), se descartaron memorias ``Gamer'' en favor de
módulos con ECC (Error Correction Code) real. En simulaciones matemáticas largas, un solo
``bit flip'' (error de memoria por radiación cósmica o interferencia) puede corromper
semanas de calculo. Esta memoria detecta y corrige esos errores en tiempo real, garantizando
que los resultados de los algoritmos sean matemáticamente precisos.

Esto se logra gracias al mecanismo SECDED (Single Error Correction, Double Error Detection),
cuando el procesador escribe datos en la RAM, un algoritmo matemático (basado en el Código 
Hamming) calcula un valor de control para esos datos y los guarda en los bits extra, cada vez
que se lee ese dato, la memoria vuelve a calcular el código y lo compara con el original, y 
si el código no coincide porque un bit cambio, el algoritmo tiene suficiente información 
matemática para saber exactamente que bit fallo y lo voltea de regreso a su estado correcto 
en mili-segundos. En caso de que fallen dos bits al mismo tiempo (algo estadísticamente muy
raro), la memoria no puede corregirlos, pero detecta el error y ordena al sistema apagarse
de forma inmediata.

\subsection{Motherboard:}
ASRock X870 Taichi Creator, es una placa base de grado ``Creator/Workstation''.

\paragraph{Conectividad:} Incluye puertos Ethernet de 10 Gb y 5Gb, permitiendo la 
transferencia de grandes datasets a servidores NAS o clústeres de computo sin cuellos de
botella de red.

\paragraph{Estabilidad:} Sus fases de poder (VRM) están sobredimensionadas para mantener al 
Ryzen 9 operando al 100\% de carga durante días sin sobrecalentarse.

\subsection{Almacenamiento:}
Kigston KC3000 2TB (NVMe Gen 4), se priorizo la durabilidad y la consistencia. Utiliza 
memoria TLC con DRAM Cache dedicada. Esto es crucial para la compilación de software y el
uso de memoria virtual (Swap), donde debe manejar miles de pequeños archivos simultáneamente
sin perder velocidad, algo que los discos económicos (QLC/DRAM-less) no pueden hacer.

\subsection{GPU:}
NVIDIA GeForce RTX 5070 Ti 16 GB, se selecciono NVIDIA por su plataforma CUDA (Compute
Unified Device Architecture) que le permite al procesador delegar tareas matemáticas pesadas
a la tarjeta gráfica, y por sus librerías que son dominantes en la ciencia de datos, machine
learning y simulación física. Los 16 GB de VRAM permiten cargar modelos tamaño considerable
en la memoria de la tarjeta para procesamiento paralelo masivo.

\subsection{Refrigeración}
Noctua NH-D15. Fiabilidad mecánica absoluta. En un entorno de trabajo crítico, se evitó la 
refrigeración líquida (que tiene riesgo de fallo de bomba o fugas a largo plazo). El 
disipador por aire de Noctua ofrece un rendimiento térmico similar con cero riesgo de fallos
catastróficos, ideal para una máquina que debe durar años con mantenimiento mínimo. Ademas incluye
pasta térmica pre-aplicada.

\subsection{Gabinete:}
Cooler Master Elite 502. Provee el espacio físico necesario (soporta coolers de hasta 170mm 
y GPUs de 410mm) con una estética profesional y sobria, adecuada para un laboratorio o 
entorno de oficina, alejándose de la estética ``Gamer'' agresiva.

\subsection{Alimentación:}
Corsair RM850e (ATX 3.0). Eficiencia energética y seguridad. Certificación ATX 3.0 para 
manejar los picos de energía transitorios de las nuevas tarjetas gráficas de la serie 50 de 
forma nativa, protegiendo la inversión de hardware ante fluctuaciones eléctricas.

\section{Conclusiones:}
La configuración presentada para una workstation científica, se ajusta y busca dar el mejor
rendimiento posible en sus funciones con el presupuesto limitado de 3000\$, gracias a su
memoria, su procesador y su refrigeración se logra un equipo de bajo mantenimiento y que
prioriza la estabilidad, la integridad y la operación continua, con la capacidad suficiente 
para ejecutar desde simulaciones físicas y álgebra lineal avanzada hasta entrenamiento
de redes neuronales profundas.

Con un mayor presupuesto se podrían expandir la cantidad de memoria RAM, e instalar
almacenamiento NVMe Gen 5 ya que la motherboard soporta esa tecnología.

\end{document}
