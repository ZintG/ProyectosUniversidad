\documentclass[12pt]{article}
\usepackage[utf8]{inputenc}   % Soporte para tildes y ñ
\usepackage[spanish]{babel}    % Configura el idioma a español
\usepackage[margin=2.5cm]{geometry} % Configura los márgenes
\usepackage{graphicx}          % Para insertar imágenes
\usepackage{amsmath}
\usepackage{parskip} % Separa párrafos con espacio y elimina la sangría

\title{Hoja de referencias arquitectura del computador}
\author{Simón Tovar CI: 31.678.578}
\date{2026}

\begin{document}

\maketitle

\tableofcontents
\newpage

\section{CAPITULO 1}
\subsection{Prestaciones}
$${\text{Prestaciones}}_x=\frac{1}{\text{Tiempo de ejecución}_x}$$
Sean $x$ e $y$ dos equipos distintos decimos que:
$$\text{Prestaciones}_x>\text{Prestaciones}_y$$
cuando: $\text{Tiempo de ejecución}_y>\text{Tiempo de ejecución}_x$ también tenemos que:
$$\frac{\text{Prestaciones}_x}{\text{Prestaciones}_y}=n$$
donde $n$ es el numero de veces que $x$ es mas rápida que $y$.

\subsubsection{Tiempo de Ejecución}
El tiempo de ejecución se define de la siguiente manera:
$$\text{Tiempo de ejecución}=\text{Ciclos de reloj de CPU} \times \text{Tiempo de ciclo de reloj}$$
Donde $\text{Tiempo de ciclo de reloj}= \frac{1}{\text{Frecuencia de reloj}}$ por lo que también se puede escribir de la siguiente forma:
$$\text{Tiempo de ejecución}=\frac{\text{Ciclos de reloj de CPU}}{\text{Frecuencia de reloj}}$$

\subsubsection{Ciclos de reloj de CPU}
Los Ciclos de reloj de CPU se definen de la siguiente forma:
$$\text{Ciclos de reloj de CPU}=\text{Instrucciones de un programa} \times \text{CPI}$$
Donde CPI es la media de ciclos por instrucción.

De esta manera ahora la formula de tiempo de ejecución la podemos escribir de la siguiente manera:
$$\text{Tiempo de Ejecución}=\frac{\text{N}^{\circ}\text{. de instrucciones}\times\text{CPI}}{\text{Frecuencia de reloj}}$$

\subsection{Fabricación}

\subsubsection{Costes}

$$\text{Coste por dado}=\frac{\text{coste por oblea}}{\text{dado por oblea}\times\text{factor de producción}}$$

$$\text{Dados por oblea}=\frac{\text{área de la oblea}}{\text{área del dado}}$$

$$\text{Factor de producción}=\frac{1}{(1+(\text{defectos por área}\times\text{área del dado}/2))^2}$$

\subsection{SPEC}

$$\text{Razón de SPEC}=\frac{\text{tiempo de ejecución en equipo de referencia}}{\text{tiempo de ejecución en equipo evaluado}}$$

$$\text{Media geométrica}=\sqrt[n]{\prod_{i=1}^n \text{Razón de SPEC}_i}$$

\subsection{Ecuaciones varias}

\subsubsection{Ley de Amdhal}
regla que
establece que el aumento posible de las prestaciones con una mejora 
determinada está limitado por la cantidad en que se usa la mejora. Esta es 
una versión cuantitativa de la ley de rendimiento decreciente en economía.

$$\text{Tiempo nuevo}=\text{tiempo no afectado}+\frac{\text{tiempo afectado}}{\text{cantidad de mejora}}$$

\subsubsection{MIPS}
medida de la velocidad de ejecución de un programa basada en el número de 
instrucciones. MIPS está definido como el número de instrucciones dividido 
por el producto del tiempo de ejecución por $10^6$.

$$\text{MIPS}=\frac{\text{numero de instrucciones}}{\text{tiempo de ejecución}\times10^6}$$

\paragraph{Ecuación clave}

$$\frac{\text{segundos}}{\text{programa}}=\frac{\text{instrucciones}}{\text{programa}} \times \frac{\text{ciclos de reloj}}{\text{instrucción}} \times \frac{\text{segundos}}{\text{ciclos de reloj}}$$

\end{document}
